% --------------------------------------------------------------
% This is all preamble stuff that you don't have to worry about.
% Head down to where it says "Start here"
% --------------------------------------------------------------
 
\documentclass[12pt]{article}
 
\usepackage[margin=1in]{geometry} 
\usepackage{amsmath,amsthm,amssymb,scrextend}
\usepackage{fancyhdr}
\usepackage{enumitem}
\usepackage{amsmath}
\usepackage{amssymb}
\usepackage{textcomp}
\usepackage{fancybox}
\usepackage{tikz}
\usepackage{tasks}
\pagestyle{fancy}
\usepackage[makeroom]{cancel}
\usepackage{graphicx}
\usepackage{caption}
\usepackage{mwe}
\usepackage{tikz}
\usetikzlibrary{positioning}

\newcommand{\N}{\mathbb{N}}
\newcommand{\Z}{\mathbb{Z}}
\newcommand{\I}{\mathbb{I}}
\newcommand{\R}{\mathbb{R}}
\newcommand{\Q}{\mathbb{Q}}
\renewcommand{\qed}{\hfill$\blacksquare$}
\let\newproof\proof
\renewenvironment{proof}{\begin{addmargin}[1em]{0em}\begin{newproof}}{\end{newproof}\end{addmargin}\qed}
% \newcommand{\expl}[1]{\text{\hfill[#1]}$}
 
\newenvironment{theorem}[2][Theorem]{\begin{trivlist}
\item[\hskip \labelsep {\bfseries #1}\hskip \labelsep {\bfseries #2.}]}{\end{trivlist}}
\newenvironment{lemma}[2][Lemma]{\begin{trivlist}
\item[\hskip \labelsep {\bfseries #1}\hskip \labelsep {\bfseries #2.}]}{\end{trivlist}}
\newenvironment{problem}[2][Problem]{\begin{trivlist}
\item[\hskip \labelsep {\bfseries #1}\hskip \labelsep {\bfseries #2.}]}{\end{trivlist}}
\newenvironment{exercise}[2][Exercise]{\begin{trivlist}
\item[\hskip \labelsep {\bfseries #1}\hskip \labelsep {\bfseries #2.}]}{\end{trivlist}}
\newenvironment{reflection}[2][Reflection]{\begin{trivlist}
\item[\hskip \labelsep {\bfseries #1}\hskip \labelsep {\bfseries #2.}]}{\end{trivlist}}
\newenvironment{proposition}[2][Proposition]{\begin{trivlist}
\item[\hskip \labelsep {\bfseries #1}\hskip \labelsep {\bfseries #2.}]}{\end{trivlist}}
\newenvironment{corollary}[2][Corollary]{\begin{trivlist}
\item[\hskip \labelsep {\bfseries #1}\hskip \labelsep {\bfseries #2.}]}{\end{trivlist}}
 
\setlength{\parindent}{0pt}
\begin{document}
 \settasks{
	counter-format=(tsk[r]),
	label-width=4ex
}
% --------------------------------------------------------------
%                         Start here
% --------------------------------------------------------------

\lhead{Math 475}
\chead{Homework 1}
\rhead{Meenmo Kang}

\begin{enumerate}
    %2.7.1
    \item For each of the four subsets of the two properties (a) and (b), count the number of four-digit numbers whose digits are either 1,2,3,4, or 5:
    \begin{enumerate}[label=(\alph*)]
        \item The digits are distinct.
        \item The number is even.
    \end{enumerate}
    Note that there are four problems here: $\phi$ (no further restriction), \{a\} (property(a) holds), \{b\} (property (b) holds), \{a,b\} (both properties (a) and (b) hold).\\
    
    %Sol.
    \begin{itemize}
        \item $\phi$: $5\times 5\times 5\times 5 = 5^4 = 625$ \hfill (since there is no restriction.)
        \item \{a\}: $5\times 4\times 3\times 2 = 120$ \hfill (sample without replacement)
        \item \{b\}: $5\times 5\times 5\times 2 = 250$ \hfill (last digit must be either 2 or 4)
        \item \{a, b\}: $2\times 4\times 3\times 2$ \\ 
        (last digit must be even AND sample without replacement rest of three digits)
    \end{itemize}
    
    
    
    %2
    \item How many orderings are there for a deck of 52 cards if all the cards of the same suit are together?
    \boldmath$$\underbrace{4!}_{\text{the order of suits}}\times \underbrace{13!^4}_{\text{the order within the suit}}$$
    
    %4
    \item How many distinct positive divisors does each of the following numbers have?
    \begin{enumerate}[label=(\alph*)]
        \item $3^4\times 5^2\times 7^6\times 11$\\
        \boldmath{$(4+1)\times (2+1)\times (6+1) \times (1+1)= 210$}\\
        
        \item 620 \boldmath{=$2^2\times 5^1\times 31$}\\
        \boldmath{$(2+1)\times (1+1)\times (1+1) = 12$}\\
        
        \item $10^{10}$ \boldmath{$=2^{10}\times 5^{10}$}\\
        \boldmath{$(10+1)\times (10+1) =121$}\\
    \end{enumerate}

    
    \newpage
    %5
    \item Determine the largest power of 10 that is a factor of the following numbers (equivalently, the number of terminal 0s, using ordinary base 10 representation):
    \begin{enumerate}[label=(\alph*)]
        \item 50!\\
        
        10 can be factored as 2 and 5 which are both prime numbers. By finding quotient of 50 divide by the power of 2 and 5, we can find how many 2 and 5 are multiplied for 50! in order to determine the largest power of 10 that is a factor of 50!.
        \begin{itemize}
            \item The quotient of $\frac{50}{2^1} = 25$
            \item The quotient of $\frac{50}{2^2} = 12$
            \item The quotient of $\frac{50}{2^3} = 6$
            \item The quotient of $\frac{50}{2^4} = 3$
            \item The quotient of $\frac{50}{2^5} = 1$
        \end{itemize}
        $\Rightarrow$ \text{their sum is 47.}
        
        \begin{itemize}
            \item The quotient of $\frac{50}{5^1} = 10$
            \item The quotient of $\frac{50}{5^2} = 2$
        \end{itemize}
         $\Rightarrow$ \text{their sum is 12.}\\
         
         We can observe that $100=2^{47}\cdot 5^{12}\cdot \alpha$. Therefore, the largest power of 10 that is a factor of 50! is 12.\\
         
        
        \item 1000!\\
        \begin{itemize}
            \item The quotient of $\frac{1000}{5^1} = 200$
            \item The quotient of $\frac{1000}{5^2} = 40$
            \item The quotient of $\frac{1000}{5^3} = 8$
            \item The quotient of $\frac{1000}{5^4} = 1$
        \end{itemize}
        $\Rightarrow$ \text{their sum is 249.}\\
        
        Therefore, the largest power of 10 that is a factor of 1000! is 249.
        
    \end{enumerate}
    
    %6
    \newpage
    \item How many integers greater than 5400 have both of the following properties?
    \begin{enumerate}[label=(\alph*)]
        \item The digits are distinct.
        \item The digits 2 and 7 do not occur.\\
    \end{enumerate}
    
    \begin{itemize}
        \item 4-digit
        \begin{itemize}
            \item If the first digit is 5
            \begin{itemize}
                \item If the second digit is 4\\
               \boldmath$\Rightarrow 6\times 5=30$
                \item Else
                \begin{itemize}
                    \item 2nd digit: All possible options are \{6,8,9\}
                    \item 3rd and 4th digit: 6 and 5 respectively\\
                    \boldmath{$\Rightarrow\; 3\times 6\times 5$}
                \end{itemize}
            \boldmath{$\Rightarrow 90$}\\
            \end{itemize}
            
            \item Else
            \begin{itemize}
                \item 1st digit: All possible options are \{6,8,9\}
                \item 2nd digit: Remaining number of digits is 7, excluding 2,7, and one of \{6,8,9\}
                \item 3rd and 4th digit: 6 and 5 respectively\\
                \boldmath{$\Rightarrow\; 3\times 7\times 6\times 5 = 630$}
            \end{itemize}
        \end{itemize}
        
        \item 5-digit\\
        \boldmath{$\Rightarrow\; 7\times 7\times 6\times 5\times 4= 5880$}\\
        
        \item 6-digit\\
        \boldmath{$\Rightarrow\; 7\times 7\times 6\times 5\times 4\times 3= 17640$}\\
        
        \item 7-digit\\
        \boldmath{$\Rightarrow\; 7\times 7\times 6\times 5\times 4\times 3\times 2= 35280$}\\
        
        \item 8-digit\\
        \boldmath{$\Rightarrow\; 7\times 7\times 6\times 5\times 4\times 3\times 2\times 1= 35280$}\\
    \end{itemize}
    $$30+90+630+5880+17640+35280+35280$$
    \\
    
    %7
    \item In how many ways can four men and eight women be seated at a round table if there are to be two women between consecutive men around the table?\\
    \boldmath{$3!\times 8!$}    
    
    
    %9
    \item In how many ways can 15 people be seated at a round table if B refuses to sit next to A? What if B only refuses to sit on A's right?
    
    \begin{itemize}
        \item Suppose A sit first at the table. Then there are 14! ways of 15 people to be seated in total.
        \item Suppose B is sitting on the left of A. Then there are 13! ways of other 13 people to be seated.
        \item Suppose B is sitting on the right of A. Then there are also 13! ways of other 13 people to be seated.
        \item As a result, if B refuses to sit next to A, there are \boldmath{$14! - 2\times 13!$} ways that 15 people can be seated at the table.\\
        
        \item If B refuses to be seated next to A, then one of other 13 people is sitting there. \item Then remaining 13 seats can be arranged as many as 13! ways.
        \item Thus there are \boldmath{$13\times 13!$} ways 15 people to be seated if B refuses to sit next to A.
    \end{itemize}
    
    \vspace{2\baseslineskip}
    %14
    \item A classroom has two rows of eight seats each. There are 14 students, 5 of whom always sit in the front row and 4 of whom always sit in the back row. In how many ways can the students be seated?\\
    
    \begin{itemize}
        \item 5 of whom always sit in the front row\\
        \boldmath{$\binom{8}{5}\times 5!$}
        
        \item 4 of whom always sit in the back row\\
        \boldmath{$\binom{8}{4}\times 4!$}
        
        \item Rest of students\\
        \boldmath{$\binom{7}{5}\times 5!$}\\
        
    \boldmath{$\Rightarrow \binom{8}{5}\times 5!\times \binom{8}{4}\times 4!\times \binom{7}{5}\times 5!$}
    \end{itemize}
    
    \newpage
    %12
    \item A football team of 11 players is to be selected from a set of 15 players, 5 of whom can play only in the backfield, 8 of whom can play only on the line, and 2 of whom can play either in the backfield or on the line. Assuming a football team has 7 men on the line and 4 men in the backfield, determine the number of football teams possible.
    \begin{itemize}
        \item  When none of those two players who can play either in the backfield or on the line are included.
        $$\binom{8}{7}\times \binom{5}{4}$$
        \item When one of those two players are included.
        $$\binom{8}{6}\times\binom{5}{4}+\binom{8}{7}\times\binom{5}{3}$$
        \item When both players are included.
        \begin{itemize}
            \item When both players play in the backfield
            $$\binom{8}{7}\times \binom{5}{2}$$
            \item When both players play on the line
            $$\binom{8}{5}\times \binom{5}{4}$$
            \item Otherwise
            $$2\left[\binom{8}{6}\times \binom{5}{3}\right]$$
        \end{itemize}
        Thus, the number of football teams possible is
        $$\binom{8}{7}\times \binom{5}{4} + \left[\binom{8}{6}\times\binom{5}{4}+\binom{8}{7}\times\binom{5}{3}\right]$$
        $$+\left\{\binom{8}{7}\times \binom{5}{2}+\binom{8}{5}\times \binom{5}{4}+ 2\left[\binom{8}{6}\times \binom{5}{3}\right] \right\}$$
    \end{itemize}

    \newpage
    %13
    \item There are 100 students at a school and three dormitories, A, B, and C, with capacities 25, 35 and 40, respectively.
    \begin{enumerate}[label=(\alph*)]
        \item How many ways are there to fill the dormitories?
        $$\binom{100}{25}\times \binom{75}{35} = \frac{100!}{25!35!40!}$$
        \item Suppose that, of the 100 students, 50 are men and 50 are women and that A is an all-men's dorm, B is an all-women's dorm, and C is co-ed. How many ways are there to fill the dormitories?
        $$\binom{50}{35}\times\binom{50}{25}$$
    \end{enumerate}
    
    %20
    \item Determine the number of circular permutations of \{0,1,2, ... ,9\} in which 0 and 9 are not opposite. (Hint: Count those in which 0 and 9 are opposite.)
    \begin{itemize}
        \item The total number of permutation: \boldmath$9!$
        \item The number of permutation in which 0 and 9 are opposite: \boldmath$8!$\\
        (Since if we fix the location of 0 and 9 in which those are opposite, the number of permutation regarding remaining seats is 8!.)
    \end{itemize}
    Thus, the number of circular permutations of \{0,1,2, ... ,9\} in which 0 and 9 are not opposite is 
    $$9!-8!$$
    %39
    \item There are 20 identical sticks lined up in a row occupying 20 distinct places as follows:
    $$|\;|\;|\;|\;|\;|\;|\;|\;|\;|\;|\;|\;|\;|\;|\;|\;|\;|\;|\;|$$
    Six of them are to be chosen.
    \begin{enumerate}[label=(\alph*)]
        \item How many choices are there?
        $$\binom{20}{6}$$
        \item How many choices are there if no two of the chosen sticks can be consecutive?\\
        
        If there are no sticks that are chosen consecutively, there must be seven partitions in total. Then consider $\{a_i\}_{i=1}^6$ to be chosen sticks and $\{x_i\}_{i=1}^7$ to be the number of bars before $a_n$. It is clear that $\{x_i\}_{i=2}^6$ are at least 1. As for $x_1$ and $x_7$, those could be 0, if the very first or last stick is chosen.\\
        
        Suppose $\{x_i\}_{i=2}^6 = \{y_i\}_{i=2}^6+1$ and $\sum_{i=1}^7 x_i = 14$ (since 6 sticks are already chosen). Then $\sum_{i=1}^7 x_i = x_1 + (y_2+1) + ... + (y_6 +1) + x_7 = 14 \Leftrightarrow$\\
        $x_1+y_2+...+y_6+x_7 = 14-5 = 9$. Since each of $x_1+y_2+...+y_6+x_7$ is at least 0, non negative integer, we can achieve the answer as below by the formula from the Theorem 2.5.1 on the textbook.
        $$\binom{9+7-1}{9}$$
        
        \item How many choices are there if there must be at least two sticks between each pair of chosen sticks?\\
        
        Since there must be at least two sticks between each pair of chosen sticks, we can consider this $\{x_i\}_{i=2}^6 = \{y_i\}_{i=2}^6+2$. Then $x_1+y_2+...+y_6+x_7 = 14-10 = 4$. Therefore, we are getting the answer as below.
        $$\binom{4+7-1}{4}$$
    \end{enumerate}
\end{enumerate}
\end{document}